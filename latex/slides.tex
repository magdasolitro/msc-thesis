%% Le lingue utilizzate, che verranno passate come opzioni al pacchetto babel. Come sempre, l'ultima indicata sarà quella primaria.
%% Se si utilizzano una o più lingue diverse da "italian" o "english", leggere le istruzioni in fondo.
\def\thudbabelopt{english,italian}
%% Valori ammessi per target: bach (tesi triennale), mst (tesi magistrale), phd (tesi di dottorato).
%% --- Beamer ---
%% La chiave "beamer" attiva la modalità slide e specifica le opzioni da passare all'omonima classe.
%% Le opzioni specificate "noamsthm,10pt" sono solamente indicative.
%% L'uso dell'opzione "ignorenonframetext" può creare problemi ed è quindi sconsigliato.
%% Se incontrate problemi con altre opzioni di "beamer" fatecelo sapere.
\documentclass[beamer={noamsthm,10pt},target=mst]{thud}[2014/03/11]

%% --- Informazioni sulla tesi ---
%% Per tutti i tipi di tesi
\title{Retrospettiva antero-posteriore dei linguaggi funzionali imperativi}
\author{Nello Sconforto}
\course{Informatica}
\supervisor{Prof.\ Sara Cinesca}
\cosupervisor{Dott.\ Remo Mori\and Dott.\ Dario Lampa}
%% Altri campi disponibili: \tutor, \date (anno accademico, calcolato in automatico).
%% Con \supervisor, \cosupervisor e \tutor si possono indicare più nomi separati da \and.
%% Per le sole tesi di dottorato
\cycle{XXVIII}
%% Campi obbligatori: \title, \author e \course.

%% Nel resto del preambolo potete personalizzare la presentazione liberamente.
%% Se incontrate problemi causati dall'interazione tra "thud" e "beamer" fatecelo sapere.

\begin{document}

%% Il frontespizio prima di tutto!
%% \maketitle accetta gli stessi argomenti opzionali di \begin{frame} (p.es. "[plain]")
\maketitle

%% Sezione
\section{Sezione}

%% Diapositiva
\begin{frame}{Diapositiva}
\begin{itemize}
\item Lorem ipsum dolor sit amet, consectetur adipiscing elit.
\item Morbi varius neque eu erat dictum pulvinar.
\item Cras pellentesque lacus sit amet magna dictum, ut tempus eros consequat.
\end{itemize}
\end{frame}

%% Sottosezione
\subsection{Sottosezione}
\begin{frame}{Altra diapositiva}
\begin{enumerate}
\item Nunc ornare elit et pretium commodo.
\item Aenean a urna eu diam interdum sagittis faucibus non quam.
\item Mauris eget massa eget ipsum condimentum varius non et eros.
\end{enumerate}
\end{frame}

\end{document}
